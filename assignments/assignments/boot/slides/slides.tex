\ifdefined\HANDOUT
  \documentclass[aspectratio=169,12pt,handout]{beamer}
  \newcommand{\answer}[1]{}
\else
  \documentclass[aspectratio=169,18pt]{beamer}
  \newcommand{\answer}[1]{#1}
\fi

\usepackage[T1]{fontenc}
%\usepackage[latin1]{inputenc}
\usepackage[utf8]{inputenc}
\usepackage{lmodern}

\usepackage{calc}
\usepackage{bytefield}

\usepackage{changepage}

\usepackage{listings} \lstset{
  language=c, 
  basicstyle=\ttfamily\small,
  showstringspaces=false
}

\usepackage{caption}

\usepackage{tikz}
\usetikzlibrary{arrows}
\usetikzlibrary{automata}
\usetikzlibrary{topaths}
\usetikzlibrary{calc}
\usetikzlibrary{positioning}

\usepackage{syntax}
\grammarindent=2cm

\let\Tiny=\tiny

\usetheme{Warsaw}

\author{Johan Montelius}
\institute{KTH}
\date{2018}

\setbeamertemplate{footline}[frame number]

\title[\course]{Memory}

\begin{document}

% For every picture that defines or uses external nodes, you'll have to
% apply the 'remember picture' style. To avoid some typing, we'll apply
% the style to all pictures.
\tikzstyle{every picture}+=[remember picture]


\begin{frame}
\titlepage
\end{frame}


\begin{frame}[fragile]{The Master Boot Record}

\begin{lstlisting}[language={[x86masm]Assembler}]
bits 16
start:
	mov ax, 0x07C0		; 0x07c00 is were we are
	add ax, 0x20		; add 0x20 (when shifted 512)
	mov ss, ax              ; set the stack segment 
	mov sp, 0x1000          ; set the stack pointer (+ 4096)

	mov ax, 0x07C0		; Set data segment
	mov ds, ax              ;
	mov si, msg		; a pointer to the message
	mov ah, 0x0E		; id of the print char procedure
\end{lstlisting}
  
\end{frame}


\begin{frame}{The workhorse: 8086}

  \begin{columns}
    \begin{column}{0.5\textwidth}
  \begin{figure}[]
    \centering
    \includegraphics[scale=0.4]{shuttle.jpg}
  \end{figure}
    \end{column}
    \begin{column}{0.5\textwidth}
      {\bf Intel 8086}
      \begin{itemize}
        \item 1978, 5 MHz
        \item 16 bit address space (64 Kbyte)
        \item 20 bit memory bus (1 Mbyte)
        \item no protection of segments
        \item segments for: code, data, stack, extra
      \end{itemize}
    \end{column}
  \end{columns}

\end{frame}


\begin{frame}{Segment addressing in 8086 - real mode}

\begin{columns}
 \begin{column}{0.5\textwidth}
  \begin{tikzpicture}
    \draw[] (2,0) rectangle (4,6);
    \draw[] (2.05,5.05) rectangle (3.95,5.95);
    \draw[] (2.05,3.05) rectangle (3.95,3.95);
    \draw[] (2.05,0.96) rectangle (3.95,1.95);
    
    \node[anchor=south, font=\small] at (6.2,5.1) {segment 16 bits};
    \draw[] (5.4,5) rectangle (7.0,5.2);
    \draw[dashed] (7.0,5) rectangle (7.4,5.2);
    
    \draw[dashed] (6.2,5) -- (6.2,4.6);
    \draw[dashed,->] (6.2,3.8) -- (6.2,3.3);


    \node[anchor=south, font=\small] at (6.6,4.1) {offset 16 bits};
    \draw[dashed] (5.4,4) rectangle (5.8,4.2);
    \draw[] (5.8,4) rectangle (7.4,4.2);
    \draw[dashed,->] (6.6,4) -- (6.6,3.3);

    \node[anchor=north, font=\small] at (6.4,2.4) {bus 20 bits};
    \draw[] (5.4,3) rectangle (7.4,3.2);

    \draw[dashed,->] (6.4,3) -- (6.4,1.6) -- (4.4,1.6);

    \node[anchor=east, font=\tt] at (2,6) {0 KB};
    \node[anchor=east, font=\tt] at (2,5) {64 KB};
    \node[anchor=east, font=\tt] at (2,4) {128 KB};
    \node[anchor=east, font=\tt] at (2,3) {192 KB};
    \node[anchor=east, font=\tt] at (2,2) {256 KB};
    \node[anchor=east, font=\tt] at (2,1) {320 KB};
    \node[anchor=east, font=\tt] at (2,0) {384 KB};
  \end{tikzpicture}           
 \end{column} 
 \begin{column}{0.5\textwidth}
   \begin{itemize}
   \item<2-> Segment register chosen based on instruction: {\em code
       segment}, {\em stack segment}, {\em data segment} (and the {\em
       extra segment}.
   \item<3-> The segment architecture available still today in {\em
       real mode} i.e. the 16-bit mode that the CPU is initally in.
   \end{itemize}
 \end{column}
\end{columns} 
\end{frame}

\begin{frame}{Segment addressing in 8086 - real mode} 


  \begin{tikzpicture}

    \node<1->[anchor=south] at (2,7) {MMU};    
    \draw<1->[dashed] (1,0) rectangle (13,6.5);

    \draw<1->[->] (0,5) -- node[midway, anchor=south west] {offset, 16 bits} (8,5) -- (8,3.1);

    \draw<2->[] (8,2.75) circle (0.3cm) node {+};
    
    \node<1->[anchor= south] at (2.5,3.7) {segment selectors};

    \draw<1->[] (1.5,3) rectangle node[] {code} (2.5,3.5);
    \draw<1->[] (1.5,2.3) rectangle node[] {stack} (2.5,2.8);
    \draw<1->[] (1.5,1.6) rectangle node[] {data} (2.5,2.1);
    \draw<1->[] (1.5,0.9) rectangle node[] {extra} (2.5,1.4);
    
    \draw<2->[->] (2.6,1.85) to[out=0, in=180] node[at end, anchor=north west, midway] {base 16-bits, shifted 4 bits} (7.6,2.75);

    \draw<2->[->] (8.5,2.75) -- (14,2.75) node[midway, anchor=south] {physical address, 20 bits};    

    


  \end{tikzpicture}
  
\end{frame}



\begin{frame}[fragile]{The Master Boot Record - cont}

\begin{lstlisting}[language={[x86masm]Assembler}]
.next:
	lodsb			; 
	cmp al, 0               ; 
	je .done		; 
	int 0x10		; 
	jmp .next               ; 
.done:
	jmp $			; loop forever

msg:	db 'Hello', 0 		; the string
	times 510-($-$$) db 0	; zeros to fill up the 512 bytes
	dw 0xAA55		; master boot record signature 
\end{lstlisting}
\end{frame}

\begin{frame}{Our MBR in memory}
  
\vspace{20pt}

\begin{tikzpicture}
  \draw<1->[] (0.9,0.5) rectangle (11.1,3);
  \node<1->[] at (8,3.5) {20 bit address space, 1Mi byte};
  \draw<2->[] (1.0,0.6) rectangle ++(0.4,2.3) node[midway, rotate=90, align=center] {BIOS};    
  \draw<4->[] (1.5,0.6) rectangle ++(2,2.3) node[midway,  align=center] {free};  
  \draw<3->[] (3.6,0.6) rectangle ++(0.4,2.3) node[midway, rotate=90, align=center] {MBR};  
  \draw<4->[] (4.1,0.6) rectangle ++(4,2.3) node[midway, align=center] {free, 600 Ki Byte};  
  \draw<2->[] (8.2,0.6) rectangle (11,2.9) node[midway, rotate=90, align=center] {BIOS};

\end{tikzpicture}

\vspace{40pt}
\pause {\em not drawn to scale}

\end{frame}


\begin{frame}[fragile]{our first kernel}

\begin{lstlisting}[language={[x86masm]Assembler}]
bits 32

global start

section .text

start:
	mov dword [0xb8000], 0x2f4b2f4f     ; print `OK` to screen
	hlt
\end{lstlisting}

\begin{verbatim}
> nasm -f elf32 kernel.asm
\end{verbatim}

  
\end{frame}

\begin{frame}[fragile]{the multiboot header}

\begin{lstlisting}
section .multiboot_header

magic   equ 0xe85250d6      ; multiboot 2
arch    equ 0               ; protected mode i386

header_start:
      dd magic                        ; magic number
      dd arch                         ; architecture
      dd header_end - header_start    ; header length 
      dd 0x100000000 - (magic + arch + (header_end - header_start))

      dw 0                 ; type
      dw 0                 ; flags
      dd 8                 ; size
header_end:
\end{lstlisting}

\begin{verbatim}
> nasm -f elf32 multiboot_header.asm 
\end{verbatim}

\end{frame}


\begin{frame}{Our kernel in memory}
  
\vspace{20pt}

\begin{tikzpicture}
  \draw<1->[] (0.9,0.5) rectangle (11.1,3);
  \draw<2->[dashed] (4.9,0) -- (4.9,3.5);
  \node<3->[] at (3,3.5) {low memeory < 1M};
  \node<4->[] at (8,3.5) {high memeory > 1M};
  \draw<5->[] (3.5,0.6) rectangle (4.8,2.9) node[midway, rotate=90, align=center] {BIOS};
  \draw<5->[] (1.0,0.6) rectangle (1.4,2.9) node[midway, rotate=90, align=center] {BIOS};    

  \draw<6->[] (1.5,0.6) rectangle (3.4,2.9) node[midway, rotate=90, align=center] {GRUB};

  \node<7->[] at (8,1.8) {free};

  \draw<8->[] (5.1,0.6) rectangle (6.4,2.9) node[midway, rotate=90, align=center] {our\\kernel};  

\end{tikzpicture}

\vspace{40pt}
\pause {\em on a x86, 32-bit protected mode}

\end{frame}


\begin{frame}[fragile]{linking}

\begin{lstlisting}
ENTRY(start)

SECTIONS {
    . = 1M;

    .boot :
    {
        *(.multiboot_header)
    }

    .text :
    {
        *(.text)
    }
}
\end{lstlisting}

\begin{verbatim}
> ld -m elf_i386 -o kernel.bin -T linker.ld multiboot_header.o kernel.o
\end{verbatim}

\end{frame}



\begin{frame}[fragile]{booting our kernel}

\begin{columns}
 \begin{column}{0.5\textwidth}
\begin{lstlisting}
cdrom/boot/
       grub/
         grub.cfg
       kernl.bin
\end{lstlisting}
 \end{column}
 \begin{column}{0.5\textwidth}
\begin{lstlisting}
set timeout=10
set default=0

menuentry "my os" {
    multiboot2 /boot/kernel.bin
    boot
}
\end{lstlisting}
 \end{column}
\end{columns}

\begin{verbatim}
> grub-mkrescue -o cdrom.iso cdrom
 :
> qemu-system-x86_64 -cdrom cdrom.iso
 :
\end{verbatim}
\end{frame}

\begin{frame}[fragile]{Let's do some C coding}

  Create a file: {\tt myos.c}
  
\begin{lstlisting}
#define COLUMNS 80
#define ROWS 25

char *name = "My OS is running";
  
typedef struct vga_char {
  char character;
  char colors;
} vga_char;
\end{lstlisting}
\end{frame}

\begin{frame}[fragile]{Let's do some C coding}

  \begin{lstlisting}
void main(void)  {

  // The VGA teminal mapped memory 
  vga_char *vga = (vga_char*)0xb8000; 	

  for (int i = 0; i < COLUMNS*ROWS; i++) {
    vga[i].character = ' ';
    vga[i].colors= 0x0f;
  }
  
  for (int i = 0; name[i] != '\0'; i++) {
    vga[600+i].character = name[i];
    vga[600+i].colors = 0x0f;
  }
  return;
}
  \end{lstlisting}

\begin{verbatim}
> gcc -m32 -c myos.c
\end{verbatim}
  
\end{frame}

\begin{frame}[fragile]{Change the kernel.asm}

  \begin{lstlisting}
bits 	32			
	
global	start	
extern	main
	
section .text	
	
start:
	cli
	mov esp, stack
        call main
	hlt
	
section .bss
	
resb 	8192
stack:		
  \end{lstlisting}
\end{frame}

\begin{frame}[fragile]{Change the linker.ld}

  Add these sections to {\tt linker.ld}
  
  \begin{lstlisting}

    .data : {
       *(.data)
       }
    .bss  : {
       *(.bss)
       }

  \end{lstlisting}
\end{frame}



\end{document}


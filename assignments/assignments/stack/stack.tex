\documentclass[a4paper,11pt]{article}

\usepackage[T1]{fontenc} 
\usepackage[latin1]{inputenc}

\usepackage{listings} \lstset{language=c, showstringspaces=false}

\usepackage{graphicx}


\newcommand{\nnsection}[1]{
\section*{#1} \addcontentsline{toc}{section}{#1} }

\begin{document}

\begin{center} \vspace{20pt} \textbf{\large A heap, a stack, a bottle and a rack.}\\
\vspace{10pt} \textbf{Johan Montelius}\\ \vspace{10pt} \textbf{HT2017}
\end{center}

\nnsection{Introduction}

In this assignment you're going to investigate the layout of a process;
where are the different areas located and which data structures goes
where.

\section{Where's da party?}

So we will will first take a look at the code segment and we will do
so using a GCC extension to the C language. Using this extension you
can store a {\em label} in a variable and then use it as any other value.

\subsection{the memory map}

Start by writing and compiling this small C program, {\tt code.c}. As
you see we use two constructs that you might not have seen before, the
label {\tt foo:} and the conversion of the label to a value in {\tt
  \&\&foo}. If everything works fine you should see the address of the
{\tt foo} label printed in the terminal. The program will then hang
waiting for a keyboard input (actually anything on the standard input
stream).


\begin{lstlisting}
#include <stdio.h>

int main() {

 foo:

  printf("the code: %p\n", &&foo);  

  fgetc(stdin);
  return 0;
  
}
\end{lstlisting}

So we have the address of the {\tt foo} label, it could be something
that looks like {\tt 0x40052a} (but could also be {\tt 0x55d8ef4426ce}
depending on which version of Linux you're running). Hit any key to
allow the program to terminate and then we try something else.

We will now tell the shell to start the process in the
background. That will allow us to use the shell while the program is
suspended waiting for input.

\begin{verbatim}
> ./code&
[1] 2708
the code: 0x55968c40f6ce
\end{verbatim}

The number $2708$ (or whatever you see) is the process identifier of
the process. Now we take a look in the directory {\tt /proc} where we find a
directory with the name of the process identifier. In this directory
there are about fifty different files and directories but the only
one we are interested in is the file {\tt maps}. Take a look at it
using the {\tt cat} command \footnote{OSX users can try: {\tt usr/bin/vmmap 2708}}.

\begin{verbatim}
$ cat /proc/2708/maps
   :
   :
\end{verbatim}

What you see is the memory mapping of the process. Can you locate the
code segment? Does it correspond to the address we found looking at the
{\tt foo} label? What is the protection of this segment?

To bring the suspended process back to the foreground you use the {\tt
  fg} command. Then you can hit any key to allow the program to
terminate.

\begin{verbatim}
 > fg
./code

>
\end{verbatim}


\subsection{code and read only data}

To make things a bit more convenient we extend our program so that it
will get its own process identifier and then print out the content of
{\tt proc/<pid>/maps}. We do this using a library procedure {\tt
  system()} that will read a command from a string and then execute it
in sub-process. 

\begin{lstlisting}
#include <stdlib.h>
#include <stdio.h>
#include <sys/types.h>
#include <unistd.h>

char global[] = "This is a global string";

int main() {

  int pid = getpid();
foo:
  printf("process id: %d\n", pid);
  printf("global string: %p\n", &global);
  printf("the code: %p\n", &&foo);    
  
  printf("\n\n /proc/%d/maps \n\n", pid);
  char command[50];
  sprintf(command, "cat /proc/%d/maps", pid);
  system(command);  

  return 0;
}
\end{lstlisting}

Look at the address of the global string, in which segment do you find
it? What is the protection of this segment, does it make sense? Now
try to allocate a global constant data structure, print it's address
and try to locate it - hmmm, is it where you thought it would be?


\begin{lstlisting}
  const int read_only = 123456;
    :
    :
    
  printf("read only:  %p\n", &read_only);
    :  
\end{lstlisting}

\section{The stack}

The stack in C/Linux is locate almost in the top of the user space and
grows downwards. You will find it in the process map and the entry
will probably look something like this:

\begin{verbatim}
7ffed89d4000-7ffed89f5000 rw-p 00000000 00:00 0          [stack]
\end{verbatim}

Before we start to play around with the stack we ponder the memory
layout of the process and how it is related to the x86 architecture. 


\subsection{the address space of a process}
  
This is on a 64-bit x86 machine so the addresses are 8 bytes wide or
16 {\em nibbles} (4-bit unit that correspond to one hex digit). Take a
look at the address {\tt 0x7ffed89f5000}, how many bits are used? What is
the 48'th bit?

In a x86-64 architecture an address field is 64-bits but only 48 are
used. The uppermost bits (63 to 47) should all be the same; in general, if they
are $0$ it's the user space and if they are $1$ it's kernel space. The
user space thus ends in:

\begin{verbatim}
 0x00 00 7f ff ff ff ff ff
\end{verbatim}


The kernel space then starts in the logical address:

\begin{verbatim}
 0xff ff 80 00 00 00 00 00
\end{verbatim}

Take a look at the memory map of our test program - the last row is a
segment that belongs to kernel space but is mapped into the user
space. This is a special trick used by Linux to speed up the execution
of some system calls - system calls that will be completed immediately
and can make no harm can be executed directly without doing a full
context swatch between user mode and kernel mode execution.

\begin{verbatim}
ffffffffff600000-ffffffffff601000 r-xp 00000000 00:00 0 [vsyscall]
\end{verbatim}

The use of the {\tt vsyscall}, and the {\tt vsdo} (which is the more
recent approach in how to improve system calls), is nothing you should
memorize but it's fun to look at things and understand where they come
from.

\subsection{back to the stack}

So we have a stack and it should be no mystery in what data structures
that are allocated on the stack. We extend our program with a data
structure that is local to the main procedure and tracks where it is
allocated in memory.

\begin{lstlisting}
#include <stdlib.h>
#include <stdio.h>
#include <sys/types.h>
#include <unistd.h>

int main() {

  int pid = getpid();
  
  unsigned long p = 0x1;

  printf("  p (0x%lx): %p \n",p, &p);

  printf("\n\n /proc/%d/maps \n\n", pid);
  char command[50];
  sprintf(command, "cat /proc/%d/maps", pid);
  system(command);  

  return 0;
}
\end{lstlisting}

Execute the program and verify that the location of {\tt p} is
actually in the stack segment. If you run the programs several times
you will notice that the stack segment moves around a bit, this is by
intention; it should be harder for a hacker to exploit a stack
overflow and change things in the heap or vice versa.

If you want the stack segment to stay put you can try to give the
following command in the shell:

\begin{verbatim}
 $ setarch $(uname -m) -R bash
\end{verbatim}

You don't have to understand what's going on but we're opening a new
bash shell where we have set the {\tt --addr-no-randomize} flag.  This
will tell the system to stop fooling around and let the data segments
be allocated where they should be allocated. Linux is trying to
protect you from hackers but we're all friends here, right. Once you
exit the shell you're back to normal.

\subsection{pushing things on the stack}

Now let's do some procedure calls and see if we can see how the stack
grows and what is actually placed on it. We keep the address of {\tt
  p}, make some calls and then print the content from the location of 
another local variable and the address of {\tt p}.

The procedure {\tt zot()} is the procedure that will do the
printing. It requires an address and will then print one line per item
on the stack. We should maybe check that the given address is higher
then the address of the local variable {\tt r} but it's more fun
living on the edge.

\begin{lstlisting}
void zot(unsigned long *stop ) {

  unsigned long r = 0x3;
  
  unsigned long *i;
  for(i = &r;  i <= stop; i++){
    printf(" %p      0x%lx\n", i, *i);
  }
}
\end{lstlisting}

We use an intermediate procedure called {\tt foo()} that we only use
to create another stack frame.

\begin{lstlisting}
void foo(unsigned long *stop ) {
  unsigned long q = 0x2;

  zot(stop);
}
\end{lstlisting}

Now for the {\tt main()} procedure that will call {\tt foo()} and do
some additional print out of the location of {\tt p} and the return
address {\tt back}.

\begin{lstlisting}
int main() {

  int pid = getpid();
  
  unsigned long p = 0x1;

  foo(&p);

 back:
  printf("  p: %p \n", &p);
  printf("  back: %p \n",  &&back);

  printf("\n\n /proc/%d/maps \n\n", pid);
  char command[50];
  sprintf(command, "cat /proc/%d/maps", pid);
  system(command);  

  return 0;
}
\end{lstlisting}

If this works you should have a nice stack trace. The interesting
thing is now to figure out why the stack looks like it does. If you
know the general structure of a stack frame you should be able to
identify the return address after the call to {\tt foo()} and {\tt
  zot()}. You should also be able to identify the saved {\em base
  stack point} i.e. the value of the stack pointer before the local
procedure starts to add things to the stack.

You also see the local data structures: {\tt p}, {\tt q}, {\tt r} and a
copy of the address to {\tt p}.  If this was a vanilla stack as
explained in any school books, you would also be able to see the
argument to the procedures. However, GCC on a x86 architecture will
place the first six arguments in registers so those will not be on the
stack. You can add ten dummy arguments to the procedures and you will
see them on the stack.

If you do some experiments and encounter elements that can not be
explained it might be that the compiler just skips some bytes in order
to keep the stack frames aligned to a 16-byte boundary. The {\em base
  stack pointer} will thus always end with a {\tt 0}.

Try locating the value of the variable {\tt i} in the {\tt zot()}
procedure. It is of course a moving target but what is the value of
{\tt i} when the location of {\tt i} is printed?

Can you find the value of the process identifier? Convert the decimal
format to hex and it should be there somewhere.

\section{The Heap}

Ok, so we have identified the code segment, a data segment for global
data structures, a strange kernel segments and the stack segment. It's
time to take a look at the heap.

The heap is used for data structures that are created dynamically
during runtime. It is needed when we have data structures that have a
size that is only known at runtime or should survive the return of a
procedure call. Anything that should survive returning from a
procedure can of course not be allocated on the stack. In C there is
no program construct to allocate data on the heap, instead a library
call is used. This is of coarse the {\tt malloc()} procedure (and its
siblings). When malloc is called it will allocate an area on the heap
- let's see if we can spot it.

Create a new file {\tt heap.c} and cut and paste the structure of our
main procedure. Keep the tricky part the prints the memory map but now
include the following section:

\begin{lstlisting}
  char *heap = malloc(20);
  
  printf("the heap variable at: %p\n", &heap);
  printf("pointing to: %p\n", heap);      
\end{lstlisting}

Locate the location of the {\tt heap} variable, it's probably where
you would suspect it to be. Where is it pointing to, a location that
is in the heap segment? 

\subsection{free and reuse}

The problem with using the heap is not how to allocate memory but to
free the memory, only free it once and not use memory that has been
freed. The compiler will detect some obvious error but most errors
will only show up at runtime and might then be very hard to track
down. If you allocate a data structure of 20 bytes every millisecond,
and forget to free it you might run out of memory in a day or two but
if you only run small benchmarks you will never notice.

Using a pointer to a data structure that has been free has an {\em
  undefined behavior} meaning that the compiler nor the execution
need to preserve any predictable behavior. This said, we can play
around with it and see what might happen. Try the following code and
reason about what is actually happening.

\begin{lstlisting}
int main() {

  char *heap = malloc(20);
  *heap = 0x61;
  printf("heap pointing to: 0x%x\n", *heap);      
  free(heap);

  char *foo = malloc(20);
  *foo = 0x62;
  printf("foo pointing to: 0x%x\n", *foo);

  /* danger ahead */
  *heap = 0x63;
  printf("or is it pointing to: 0x%x\n", *foo);      

  return 0;
}  
\end{lstlisting}

If you experiment with freeing a data structure twice you will most
certainly run in to a segmentation fault and a core dump. The reason
is that the underlying implementation of malloc and free assume that
things are structured in a certain way and when they are not things
break. Remember that by definition the behavior is
\underline{undefined} so you can not rely on that things will crash
when you test your system.

To see what is going on (and this will differ depending on what
operating system you're using) we can look at the location just before
the allocated data structure. We will here use {\tt calloc()} that
will not only allocate the data structure but also set all its elements
to zero. Try the following, also print the memory map as before.

\begin{lstlisting}
  long *heap = (unsigned long*)calloc(40, sizeof(unsigned long));

  printf("heap[2]:  0x%lx\n", heap[2]);          
  printf("heap[1]:  0x%lx\n", heap[1]);        
  printf("heap[0]:  0x%lx\n", heap[0]);      
  printf("heap[-1]: 0x%lx\n", heap[-1]);      
  printf("heap[-2]: 0x%lx\n", heap[-2]);
\end{lstlisting}

So we're cheating and access position {\tt -1} and {\tt -2}. As you
see there is some information there. Now change the size of the
allocated data structure and see what's happening. One thing you might
wonder is how free knows the size of the object that it is about to
free - any clues?

Now look at this - we're going to free the allocated space but then we
cheat and print the the content. Add the following below the code we have above.

\begin{lstlisting}
  free(heap);

  printf("heap[2]:  0x%lx\n", heap[2]);          
  printf("heap[1]:  0x%lx\n", heap[1]);        
  printf("heap[0]:  0x%lx\n", heap[0]);      
  printf("heap[-1]: 0x%lx\n", heap[-1]);      
  printf("heap[-2]: 0x%lx\n", heap[-2]);
\end{lstlisting}

Has something changed? Is it just garbage or can you identify what it
is? Take a look at the memory map, what is happening?


\section{A shared library}

When we first printed the memory map there was of course a lot of
things that you had no clue of what they were. One after one we have
identified the segments for the code, data, strange kernel stuff,
stack and the heap. There has also been a lot of junk in the middle,
between the heap and the stack. Some of the segments are executable,
some are writable, some are described by something similar to:

\begin{verbatim}
        /lib/x86_64-linux-gnu/ld-2.23.so
\end{verbatim}

All of the segments are allocated for shared libraries, either the code
or areas used for dynamic data structures. We have been using library
procedures for printing messages, finding the process identifier and
for allocating memory etc. All of those routines are located somewhere
in these segments. The malloc procedures keep information about the
data structures that we have allocated and freed and if we mess this
up by freeing things twice of course things will break. If you do
these experiments early in the course you might not know at all what
we're talking about, but it will all become clear.

\section{Summary}

You can learn allot about how things work by running very small
experiments. You should test things for yourself and experience the
things you learn in the course. It's one thing reading about a
segmentation fault on a slide, and another experience it by writing 
a small program. Remember the golden rule of engineering:

\begin{quote}
If In Doubt, Try It Out! 
\end{quote}


\end{document}

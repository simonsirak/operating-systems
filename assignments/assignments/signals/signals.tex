\documentclass[a4paper,11pt]{article}

\usepackage[T1]{fontenc} 
\usepackage[latin1]{inputenc}

\usepackage{ifthen}

\usepackage{listings}
\lstset{language=C} 
\lstset{showstringspaces=false}

\newcommand{\nnsection}[1]{
\section*{#1}
\addcontentsline{toc}{section}{#1}
}

\begin{document}

\begin{center} \vspace{20pt} \textbf{\large Don't do this at home}\\
\vspace{10pt} \textbf{Johan Montelius}\\ \vspace{10pt} \textbf{VT2016}
\end{center}


\nnsection{Getting started}

This is a small tutorial that lets you explore signals in a Linux
environment. You need access to a Unix machine and know how to use the
C compiler. The small examples that you will implement are not rocket
science but if you have done them once, it will hopefully help you
understand the operations of the operating system and how it can control
processes.

Note - the examples that are given in this tutorial are probably
nothing that you will ever use in regular programming. Some are even
things that you should never do; the examples are here to help you
gain a better understanding of what is going on under the hood
i.e. how the operating system can interact with running programs.


\section{Signals}

Signals are primarily used by the operating system to signal to
processes that something has happened that probably needs some
attention. It could also be used in between processes or even inside a
process to raise an exception.

If you open a shell you can list all possible signals by using the
command {\tt kill}. As you see there are quite many, but don't be
afraid, you do not have to learn them by heart.

\begin{verbatim}
> kill -l
\end{verbatim}

You might have used the {\tt  kill} command when you wanted  to {\em
kill} a process but the command will be able to send  any signal to a
process. When no signal is given the  {\tt SIGTERM} signal is sent to the
process. You might have learned to  write {\tt kill -9} when you really
wanted to kill something; what is signal number 9 called?

When you hit {\tt ctrl-c} in a terminal window, the shell will send a
{\tt INT} (interrupt) signal to the foreground processes. There is some
magic taking place when you're running a program in a shell and the {\tt
  ctrl-c} will terminate the program but not the shell; all will be
crystal clear ... in a couple of weeks.

To learn a bit more about how signals work we will write small
examples and see how they behave.

\section{catching a signal}

Normally when you run a program the process inherits the default
{\em signal handlers} of its creator. The process can then set its own
signal handlers to change the behavior of the process. One example is
when a process wants to do some final things before terminating when
it receives the {\tt SIGTERM} signal. 

\begin{lstlisting}
#include <stdio.h>
#include <signal.h>
#include <unistd.h>

int volatile count;

void handler(int sig) {
  printf("signal %d ouch that hurt\n", sig);
  count++;
}

int main() { 

  struct sigaction sa;

  int pid = getpid();

  printf("ok, let's go, kill me (%d) if you can!\n'', pid)

  sa.sa_handler = handler;
  sa.sa_flags = 0;
  sigemptyset(&sa.sa_mask);

  if(sigaction(SIGINT, &sa, NULL) != 0) {
    return(1);
  }
  
  while(count != 4) {
  }

  printf("I've had enough!\n");
  return(0);
}
\end{lstlisting}

Copy this program, call it test1.c, compile and execute it in a terminal. In
this terminal try to kill it by hitting {\tt ctrl-c}. The program does
not do very much besides trying to stay alive.

If we take a look at how it achieves this we see that it uses a {\tt
sigaction} structure and a call to {\tt sigaction}. The structure
holds some parameters to control what will happen. We will set the
{\tt sa\_handler} property to point to our own handler procedure. We
will then clear the {\tt sa\_mask} entry because we do not want to
block any other signals when we're in the handler. 

The interesting thing is the call to {\tt sigaction}, here we pass
three arguments: the signal we want to handle, a pointer to the
sigaction structure and a null pointer (that we don't have to bother
about now). This is the library call that will change our signal
table and when a {\tt INT} signal is sent we will be able to do what
we want.

The signal handler itself will in this example not do anything
useful, only print a message so that we know that it was invoked. Note
that one should not use a system library call such as printf inside a
handler since this might be in conflict with an ongoing library
call. We do so in this exercise to keep things as simple as possible.


Run the program and do some experiments; can you kill it by sending a
{\tt kill -2} from another terminal. What happens if you send
it a {\tt SIGTERM} signal?

Look up {\tt kill} and {\tt sigaction} using the {\tt man}
command. There is probably a lot more here than you would ever want to
know but you should get the habit of reading the man pages. 

This is the simple way to catch a signal; using a slightly different
technique we can get some more information on what is going on.


\section{catch and throw}

In Java you're quite used to declaring what exception methods could
cause and you could always trap them in an exception handler to
possibly do something else. When you're programming in C there is no
such support in the language, but it turns out that you can use signals
to handle exceptions. 

Let's catch any exception caused by division by zero. The handler will
simply print an error message and then exit but we could of course have
done something to save the situation.

\begin{lstlisting}
void handler(int sig) {
  printf("signal %d was caught\n", sig);
  exit(1); 
  return;
}
\end{lstlisting}

We need to install the handler and do so by registering it under the
{\tt SIGFPE} signal. Then we call a not so good procedure that will
divide by zero (note, it's integer division so it will generate a fault).

\begin{lstlisting}
int not_so_good() {
  int  x = 0;
  return 1 % x;
}
\end{lstlisting}

\begin{lstlisting}
int main() {
  struct sigaction sa;

  printf("Ok, let's go - I'll catch my own error.\n");  

  sa.sa_handler = handler;
  sa.sa_flags = 0;
  sigemptyset(&sa.sa_mask);

  /* and now we catch ... FPE signals */
  sigaction(SIGFPE, &sa, NULL);
  
  not_so_good();

  printf("Will probably not write this.\n");    
  return(0);
}
\end{lstlisting}

You might ask yourself how a division by zero is detected as a fault
and how it is turned into a signal to the user process. The fault is
first detected by the hardware; when the instruction is executed an
exception is raised. The CPU will then look in an {\em Interrupt
  Descriptor Table} (IDT) and jump to a location in memory that
hopefully contains some code that will take care of the problem. This
code is part of the kernel so the kernel decides what to do. In the
case of a division by zero the user process that generated the fault
must of course be interrupted. If the process has registered its own
{\tt SIGFPE} handler, as in the case above, control is passed to this
function; the default procedure is to kill the process.

\section{who do you think you are}

If we use the simple version of signal handler then there is not much
information to go on; a signal has been sent but we don't know much
more. We can however use a slightly more elaborate call that gives
us some more information. To use this version we set a flag in the
sigaction structure.

\begin{lstlisting}
int main() {

  struct sigaction sa;

  int pid = getpid();

  printf("Ok, let's go - kill me (%d) and I'll tell you who you are.\n", pid);  
  
  /* we're using the more elaborated sigaction handler */
  sa.sa_flags = SA_SIGINFO;
  sa.sa_sigaction = handler;
 
  sigemptyset(&sa.sa_mask);

  if(sigaction(SIGINT, &sa, NULL) != 0) {
    return(1);
  }
  
  while(!done) {
  }

  printf("Told you so!\n");
  return(0);
}
\end{lstlisting}

Nothing strange there, but now the handler will be passing three
arguments: the signal number, a pointer to a {\tt siginfo\_t} structure
and a pointer to a {\em context} that we can ignore for a while.

\begin{lstlisting}
void handler(int sig, siginfo_t *siginfo, void *context) {
  
  printf("signal %d was caught\n", sig);

  printf("your UID is %d\n", siginfo->si_uid);
  printf("your PID is %d\n", siginfo->si_pid);

  done = 1;
}
\end{lstlisting}

The {\tt siginfo\_t} structure contains information about the process
that sent the signal. If you start this program in one shell and then
try to kill it from another shell (using {\tt kill -2}) you should hopefully be able to
tell which process that tried to kill you.

\section{don't do this at home}

While the {\tt siginfo\_t} structure will give you more information
about why the signal was generated the {\em context} provided in the
third argument will give you more information about the environment in
which the signal was generated. 

The code below is nothing you should ever write and it's probably
something that should never be used as an example but it shows what you
can do when you go under the hood of regular C programming. We shall
do something as stupid as trying to handle a fault when an illegal
instruction is executed.

The context that is provided to a signal handler is the complete
environment of the thread that caused the fault. This environment
holds among other things the program counter i.e. in our case the
address that causes the fault. When we have done what we should in the
signal handler, control is passed back to the thread that will
start the execution on the address of the program counter. What would
happen if we help the thread a bit and advance the program counter to
the next address?

\begin{lstlisting}
#define _GNU_SOURCE  /* to define REG_RIP */

#include <stdio.h>
#include <signal.h>
#include <ucontext.h>

static void handler(int sig_no, siginfo_t* info, void *cntx)
{
  ucontext_t *context = (ucontext_t*)cntx;
  unsigned long pc = context->uc_mcontext.gregs[REG_RIP];
  
  printf("Illegal instruction at 0x%lx value 0x%x\n",pc,*(int*)pc);
  context->uc_mcontext.gregs[REG_RIP] = pc+1;
}
\end{lstlisting}  

To achieve this we define the macro {\tt \_GNU\_SOURCE} that will make
the header file {\tt ucontext.h} include a definition of the constant
{\tt REG\_RIP}. Note that we're now doing things that are hardware
dependent and we are now assuming a 64-bit Linux, if you run a 32-bit
version you could try using {\tt REG\_EIP} instead.

So now let's register the handler under the {\tt SIGSEGV} signal and
run the program into a sequence of faulty operations. Will we ever hit
the printf statement?

\begin{lstlisting}
int main()
{

  struct sigaction sa;
  sa.sa_flags = SA_SIGINFO;
  sa.sa_sigaction = handler;

  sigemptyset(&sa.sa_mask);

  sigaction(SIGSEGV, &sa, NULL);

  printf("Let's go!\n");

  /* this is nothing you should ever do */
  asm(".word 0x00000000");

 here:
  printf("Piece of cake, this call is here 0x%x!\n", &&here);
  return 0;
}
\end{lstlisting}

The example above is of course absurd, to advance the program counter
in the hope of finding an executable instruction could of course lead
us anywhere.

\section{summary}

Signals is the mechanism that the kernel uses to inform processes about
exceptions in the normal execution. If the user process has not stated
otherwise, the kernel will decide what to do. The user process can
register a {\em signal handler} to decide what to do.  The kernel will
then pass control to a specified procedure.

Signals can also be used in between processes, to send notifications to, or
control processes. The signals themselves contain no information, but
the kernel can provide more information about who sent the signal and
possibly why.


\end{document}






\documentclass[a4paper,11pt]{article}

\usepackage[T1]{fontenc} 
\usepackage[latin1]{inputenc}
 
\usepackage{listings} \lstset{language=c, showstringspaces=false, xleftmargin=-0pt, frame=l}

\usepackage{graphicx}


\newcommand*\justify{%
  \fontdimen2\font=0.4em% interword space
  \fontdimen3\font=0.2em% interword stretch
  \fontdimen4\font=0.1em% interword shrink
  \fontdimen7\font=0.1em% extra space
  \hyphenchar\font=`\-% allowing hyphenation
}
\begin{document}

\begin{center} \vspace{20pt} \textbf{\large That will be the day,...}\\
\vspace{10pt} \textbf{Johan Montelius}\\ \vspace{10pt} \textbf{HT2018}
\end{center}


\section{Introduction}

This is an assignment where you will implement your own malloc using
the buddy algorithm to manage the coalescing of free blocks. You
should be familiar with the role of the allocator and how to implement
and benchmark a simple version of it. In this presentation you will not be
given all details on how to do the implementation, but we will go
through a general strategy.


\section{The buddy algorithm}

The idea of the buddy algorithm is that given a block to free, we could
quickly find its sibling and determine if we can combine the two into
one larger block. The benefit of the buddy algorithm is that the
amortized cost of the allocate and free operations are constant. The
slight down side is that we will have some internal fragmentation
since blocks only come in powers of two.

We will keep the implementation as simple as possible, and therefore we will not
do the uttermost to save space or reduce the number of operations. The
idea is that you should have a working implementation and do some
benchmarking.


\subsection{the implementation}

Your implementation does not need to follow these guide lines but you
can take them as a starting point. Feel free to improve or implement a
different strategy (if it is equally good, better or have some
advantage).



\subsubsection{the setting}

In this implementation the largest block that we will be able to
handle is 4 Ki byte. We can then split it into sub-blocks and the
smallest block will be 32 bytes ($2^5$). This means that we will have
8 different sizes of blocks: 32, 64, 128 , .. 4 Ki. We will refer to
these as level 0 block, level 1 block etc.  We encode these as
constants in our system: {\em MIN} the smallest block is $2^{MIN}$,
{\em LEVELS} the number of different levels and {\em PAGE} the size of
the largest block ($2^{12} = 4 Ki$).

\begin{lstlisting}
#define MIN  5      
#define LEVELS 8
#define PAGE 4096
\end{lstlisting}

When we allocate a new block of 4 Ki byte, it needs to be aligned on a
4 Ki boundary i.e. the lower 12 bits are zero. This is important and
we will get this for free if we choose the system page sizes when
allocating a new block.

\subsection{operations on a block}

A block consists of a {\em head} and a byte sequence. It is the byte
sequence that we will hand out to the requester. We need to be able to
determine the size of a block, if it is taken or free and, if it is
free, the links to the next and previous blocks in the list.


\begin{lstlisting}
enum flag {Free, Taken};

struct head {
  enum   flag  status;
  short  int   level;
  struct head  *next;
  struct head  *prev;  
};
\end{lstlisting}

The size of the block is represented by the {\em level} and is encoded as
$0, 1, 2$ etc. This is the index in the set of free lists that we will
keep. Index $0$ is for the smallest size blocks i.e. 32 byte.

We will define a set of functions that do the magic of the buddy
algorithm. Given these functions we should be able to implement the
algorithm independent of how we choose to represent the blocks.

\subsubsection*{a new block}

To begin with we have to create new block. We do this using the 
  {\em mmap()} system call. This procedure will allocate new memory for our
process and here we allocate a full page i.e. a 4 Ki byte segment. Here we
take for granted that the segment returned by the {\em mmap()} function
is in fact aligned on a 4 Ki byte boundary. 


\begin{lstlisting}
struct head *new() {
  struct head *new =  (struct head *) mmap(NULL,
                                           PAGE,
                                           PROT_READ | PROT_WRITE,
                                           MAP_PRIVATE | MAP_ANONYMOUS,
                                           -1,
                                           0);
  if(new == MAP_FAILED) {
    return NULL;
  }
  assert(((long int)new & 0xfff) == 0);  // 12 last bits should be zero 
  new->status = Free;
  new->level = LEVELS -1;
  return new;
}
\end{lstlisting}

Look-up the man pages for {\em mmap()} and see what the arguments mean. When
you extend the implementation to handle larger blocks you will have to
change these parameters.

\subsubsection*{who's your buddy}

Given a block we want to find the buddy. We do this by
toggling the bit that differentiate the address of the block from its
buddy. For the 32 byte blocks, that are on level $0$, this means that
we should toggle the sixth bit. If we shift a one five positions ({\em
  MIN}) to the left we have created a mask that we can xor against the
pointer to the block.

\begin{lstlisting}
struct head *buddy(struct head* block) {
  int index = block->level;
  long int mask =  0x1 << (index + MIN);
  return (struct head*)((long int)block ^ mask);
}
\end{lstlisting}

\subsection*{split a block}

When we don't have a block of the right size we need to divide a
larger block in two. To do this we set the bit that will separate two
budies at the requested level. In the code below we do almost the
opposite of the buddy operation, we find the level of the block,
subtract one and create a mask that is or:ed with the original
address. This will now give us a pointer to the second part of the
block.

\begin{lstlisting}
struct head *split(struct head *block) {
  int index = block->level - 1;
  int mask =  0x1 << (index + MIN);
  return (struct head *)((long int)block | mask );
}
\end{lstlisting}


\subsubsection*{merge  two blocks}

We also need a function that merges two buddies. When we are to
perform a merge, we first need to determine which block is the {\em
  primary} block i.e. the first block in a pair of budies. The primary
block is the block that should be the head of the merge block so it
should have all the lower bits set to zero. We acheive this by masking
away the lower bits up to and including the bit that differentiate the
block from its buddy. Note that it does not matter which buddy we
choose, the function will always return the pointer to the primary
block.

\begin{lstlisting}
struct head *primary(struct head* block) {
  int index = block->level;
  long int mask =  0xffffffffffffffff << (1 + index + MIN);
  return (struct head*)((long int)block & mask);
}
\end{lstlisting}


\subsubsection*{the magic}

We use the regular magic trick to hide the secret head when we return a
pointer to the application. We hide the head by jumping forward one
position. If the block is in total 128 bytes and the head structure is
24 bytes we will return a pointer to the 25'th byte, leaving room for 104
bytes.

\begin{lstlisting}
void *hide(struct head* block) {
  return (void*)(block + 1);
}
\end{lstlisting}

The trick is performed when we convert a memory reference to a head structure, by
jumping back the same distance.

\begin{lstlisting}
struct head *magic(void *memory) {
  return ((struct head*)memory - 1);
}
\end{lstlisting}


\subsubsection*{level}

The only thing we have left to do is to determine which block we
should allocate. If the user request a certain amount of bytes we need
a slightly larger block to hide the head structure. We find the level
by shifting a size right, bit by bit. When we have found a size that
is large enough to hold the total number of bytes we know the level.

\begin{lstlisting}
int level(int req) {
  int total = req  + sizeof(struct head);

  int i = 0;
  int size = 1 << MIN;
  while(total > size) {
    size <<= 1;
    i += 1;
  }
  return i;
}
\end{lstlisting}

\subsubsection{before you continue}

Implement the above functions in a file called {\em buddy.c}. Also
implement a function {\em test()} that runs simple tests that ensures
you that the operations works as expected. You can for example create
a new block, divide it in two, divide in two and then hide its header,
find the header using the magic function and then merge the
blocks. Add a file {\em buddy.h} that holds a declaration of the test
procedure.

In a file {\em test.c} write a {\em main()} procedure that calls the
test procedure. Compile, link and run. Do extensive testing, if
anything is wrong in these primitive operations it will be a nightmare
to debug later.


\subsection{the algorithm}

So given that the primitive operations work as intended, it is time to implement
the algorithm. Your task is now to implement two procedures {\em
  balloc()} and {\em bfree()} that will be the API of the memory
allocator. We do not replace the regular {\em malloc()} and {\em free()}
since we want to use them when we write our benchmark program.

We will use one global structure that holds the double linked free
lists of each layer.

\begin{lstlisting}
struct head *flists[LEVELS] = {NULL};
\end{lstlisting}

The implementation can be broken down into two phases. The first phase
will hide or strip the magic head structure, and the second phase
can work on regular block structures. This could be one way to
go:

\begin{lstlisting}
void *balloc(size_t size) {
  if( size == 0 ){
    return NULL;
  }
  int index = level(size);
  struct head *taken = find(index);
  return hide(taken);
}
\end{lstlisting}


\begin{lstlisting}
void bfree(void *memory) {
  if(memory != NULL) {
    struct head *block = magic(memory);
    insert(block);
  }
  return;
}
\end{lstlisting}

It's now time for you to implement the procedures {\em find()} and {\em
  insert()}. First write down the pseudo code, what needs to be done
in each step. Draw pictures that describes how the double linked list
is manipulated. Do small test as you proceed, make sure that simple
things work before you continue - happy hacking.

\section{Benchmarking}

When your buddy allocator works as expected it's time to do some
benchmarks. The things we want to know is:

\begin{itemize}
\item the cost of a bfree and balloc operation
\item does it vary with the size of the memory
\item what is the memory utilization, free vs taken
\item ...
\end{itemize}

You will need a benchmark program that can vary the requests given a
min and max size. You should also be able to change the number of
blocks requested and the number of blocks the application has
allocated at a given time. The sequence of operations should
preferably match a real program i.e. probably not be a sequence of
identical requests.

Set up some benchmarks and try to describe the properties of your
implementation. Try to pinpoint situations where it does not perform
as well.

\section{Improving the performance}

When you have things up and running you might want to improve the
system. Before you do this, you should however get an understanding of the
costs and if it's worth improving upon. Here are some ideas that you could consider.

\subsection{the size of the head}

The size of the head structure is one thing you can take a look at. If
you have implemented it as above it is 24 bytes (4+4+8+8) which we
might improve. Since we want to hand out segments aligned by 8 bytes it's not possible to improve it by a byte or two, we have to do something radical!

One idea is that the head structure only needs to hold its two pointers if
it is a free block. Only then are the pointers needed for linking. A
taken block only need the status flag and the level counter. When we hide the
head we do not have to jump 24 bytes forward but only as far as beyond
the status and level fields. Try the following:

\begin{lstlisting}
struct peggysue {
  enum flag status;
  short int level;  
};
\end{lstlisting}

Now alter the code section that performs the magic trick, as well for the calculation of how big a block needs to be and use Peggy Sue. Since this structure is only 8
bytes we will allow more user information in the block that we
use. This might not be of significance if the block is 128 bytes to start with
but it means a lot if the block is 32 bytes.

Note, the smallest possible block is still 32 bytes since when converted to a free
block it needs to be able to hold the 24 bytes representing the head structure. 

\subsection{tagged pointers}

One way to shrink the size of the head structure is to let it hold two
pointers only. The information that tells us if it is free or not and
what level it belongs to could be encoded in bits not used by the
pointers. Our pointers will always point to a block and if these are
sixteen byte aligned that means that the four least significant bits
could be zero.

We could use the four bits of the next pointer to tell us if the
block is free or not and the four bits of the previous pointer to tell
us the level of the block. If we do this then the smallest block could
be 16 bytes instead of 32.

If you implement this you must of course mask the pointers before using
them but it could be quite easily hidden in a macro. Note however that
whenever you play around with bits, relying on that they will not be
used, will make you code less portable (but that shouldn't stop you). 

  
\subsection{return pages}

If we run large benchmarks you will have more and more 4 Ki byte pages
being allocated. This is normal since the chance of requesting many
large segments at the same time increase as we randomly choose the
size of the segments. We will never return an allocated page but you
could of course do this when a level 7 page is created from two level
6 blocks. We might want to keep one or two but why have ten lying
around?

Do a check when a full page is created and see if you already have
four pages in your free list. If that is the case you should be able
to return the page to the operating system.

\subsection{bit maps}

Handling a double linked lists is expensive; you will do several
operations and they might be reading and writing to memory locations
that are not held by the cache. A completely different approach is to
keep track of free blocks in a bit map.

Let's say that we only have one 4 Ki byte block to think about. This
could be broken down into 128 blocks of 32 bytes each. If each block
is encoded as one bit this would mean that we would only need 16 bytes
to encode all the information. We would still need to have the size
information in each block so each block needs a head structure but we
could do with out the pointers.

When we're looking for a free block of size 64 bytes, we then scan the
bit map and look for two consecutive bits that are not set. If we find
the two bits we can also determine the address of this block.

The head structure would still need 8 bytes but now we might squeeze in
some more information there. If the level is from 0 to 7, this will only require
three bits. The status flag is of course only one bit, so that would
leave us with 7 bytes to encode something else. Could we use these extra bits to
encode which bitmaps the block belongs to if we have several bitmaps?

\subsection{a larger heap}

In the implementation above we use a page as the largest possible
block. We could however extend this and use the {\em huge page} option
for {\em mmap()}. Look up the man pages and see if you can change the
code to work with pages of 2 Mi byte. It might not work out of the box
but with some configurations it should work; this command might come in handy:

\begin{verbatim}
> sudo bash -c "echo 10 > /proc/sys/vm/nr_hugepages"
\end{verbatim}


\section{Summary}

If you have completed the buddy allocator and done some benchmarking
you should have a better understanding of how {\em malloc()} and {\em
  free()} works - or rather could work; the regular malloc used by
Linux uses another protocol to keep track of free blocks.

The buddy algorithms have some advantages and some
disadvantages. Benchmarking the system will give you a good
understanding of its performance. The limitation is that we will have
some, predictable, level of internal fragmentation and that the
smallest available block is, in our implementation, quite large.


\end{document}

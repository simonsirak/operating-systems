\documentclass[a4paper,11pt]{article}

\usepackage[T1]{fontenc} 
\usepackage[latin1]{inputenc}

\usepackage{listings} \lstset{language=c, showstringspaces=false, xleftmargin=-60pt, frame=l}

\usepackage{graphicx}


\newcommand{\nnsection}[1]{
\section*{#1} \addcontentsline{toc}{section}{#1} }

\begin{document}

\begin{center} \vspace{20pt} \textbf{\large Threads - roll your own}\\
\vspace{10pt} \textbf{Johan Montelius}\\ \vspace{10pt} \textbf{VT2016}
\end{center}


\section{Introduction}

This is an experiment where we will explore the concept of an {\em
  execution context} and implement concurrency to better understand
how multi-threading works. You will basically do the first steps of
what is needed to create your own thread library. 

The program that you will write does probably not look like anything
that you will ever do again but it's fun to see that it works. After
this assignment you will have a far better understanding of what an
execution context is and how multi-threading works.


\section{An execution context}

You know that an execution context consists of a heap, a stack, code,
an instruction pointer, registers etc When you're programming in C you
only use the heap explicitly, the stack etc are implicitly used when
doing procedure calls and declaring local variables. The {\em context
  of the process} also includes open file descriptors, signal handlers
and a myriad things that the operating system keeps track of. In this
assignment we will only deal with the things that are specific for an
{\em execution} i.e. the stack and instruction pointer.


We will explore what an execution context means by the support found
in the library {\tt ucontext}. Here we find a set of procedures that
allows us to capture the current execution context, change it and even
start using another context. As you will see we will be able to switch
from one execution to another by switching context. 

\subsection{get the context}


The magic in this assignment is the data structure {\tt ucontext\_t}
that among other things holds a pointer to an execution stack, copy of
the registers and instructions pointer. We will only touch the
execution stack explicitly and let the procedures of the {\tt
  ucontext} library do the rest.

The header files might not be installed on your system from start so
first make sure that we have all the header files needed. Run the
following command in the shell:

\begin {verbatim}
>sudo apt-get install linux-headers-$(uname -r)
\end{verbatim}x

We then turn to our program. The first ting we will try, is to get
hold of the current execution context and swap one context for
another. Try the following in a file called {\tt switch.c}.


\begin{lstlisting}
#include <stdlib.h>
#include <stdio.h>
#include <ucontext.h>

int main() {

  int done = 0;
  ucontext_t one;
  ucontext_t two;

  getcontext(&one);

  printf("hello \n");
  
  if(!done) {
    done = 1;
    swapcontext(&two, &one);
  } 
  return 0;
}
\end{lstlisting}

We allocate two contexts, {\tt one} and {\tt two}, and then use the
procedure {\tt getcontext()} to populate the first context with the
state of the running context. This will store among other things the
program counter in the context data structure. We then print {\tt
  hello} to the standard output and move to the strange looking
section where we call {\tt swapcontext()}. 

The first time we enter this section {\tt done} will be zero so the
test will succeed and we will select the alternative. We set the
variable {\tt done} to one since we only want to do this once and then
call {\tt swapcontext()}. This call will save the current execution
context in structure {\tt two} and copy the context of {\tt one} into
the proper registers of the CPU. When we continue we will of course
continue from the position where {\tt one} was saved. Compile the
program and explain what you see.

\subsection{registers and optimizations}

To see that it is more than just the instruction pointer that is
stored in the context we can try the following small change to our program.

\begin{lstlisting}
  register int done = 0;
\end{lstlisting}

The {\tt register} modifier will ask the compiler to store this
variable in a register (if possible). If we now compile and run our
program something else will happen (be quick on the Ctrl-C).

To complicate things even further we can try to compile the original program
using some compile optimizations. Try the following and see how
things change, what is going on?

\begin{verbatim}
 $ gcc -O2 switch.c
 $ ./a.out 
  : 
  :
\end{verbatim}

So we learn that the compiler can do strange things. Most of the time
it does things that works, and works better, but sometime things does
not really work as we think it works. Let's not use compiler
optimizations for the rest of this assignment.

\section{allocating a stack}

We will now make the situation a bit more complicated, we will
allocate a new stack. We will have three contexts all in all. One is
the {\em main context}, that will have the regular C stack. The other
two contexts will have their own stacks that we have allocated on the
heap. We will also explicitly set the program pointers of the other
two contexts to make them start their execution on some more
interesting place. 

Create a new file called {\tt yield.c} and try the following. We will
go through it step by step. The first section is include directives
and the declarations of our contexts. We will no keep these as global
data structures since we want to make the code simpler.

\begin{lstlisting}
#include <stdlib.h>
#include <stdio.h>
#include <ucontext.h>

#define MAX 10

static int running; 

static ucontext_t cntx_one;
static ucontext_t cntx_two;
static ucontext_t cntx_main;
\end{lstlisting}

We then define a procedure {\tt yield()} that will allow us to switch
between the two contexts. We assume that the variable {\tt running} is
either {\tt 1} or {\tt 2} depending on which context that is
executing, switching between them is then a simple task. 


\begin{lstlisting}
void yield() {
  printf(`` - yield -\n'');
  if(running==1) {
    running = 2;
    swapcontext(&cntx_one, &cntx_two);
  }  else {
    running = 1;
    swapcontext(&cntx_two, &cntx_one);
  }
}
\end{lstlisting}


Next we define a procedure {\tt push()} that only serves the purpose
of making the stack grow. We do a print out when we push a call on the
stack and then another on the way down. The format string will print a
number followed by an indent ``push'' or ``pop'' where the
indentations is given by the recursion depth.

\begin{lstlisting}
void push(int p, int i) {
  if(i<MAX) {
    printf("%d%*s push\n", p, i," "); 
    push(p, i+1);
    printf("%d%*s pop\n", p, i, " "); 
  } else {
    printf("%d%*s top\n", p, i, " ");
  }
}
\end{lstlisting}

Now for the thing that will {\em tie the room together}, setting up
the two contexts and take it for a spin. We allocate two stacks,
8Kbyte each, called {\tt stack1} and {\tt stack2} (are they allocated
on the stack or on the heap?). Next we initialize each context using
the procedure {\tt getcontext()}, this will populate the contexts with
all the information that the operating system needs. Then we
explicitly change the {\tt uc\_stack} elements of the contexts. This is
a structure holding pointers and size of the stack. We are thus giving
each context a stack of their own.

We then call {\tt makecontext()} providing a reference to each context
and the procedure that they should call when starting the execution. We
provide the name of the procedure, the number of arguments and the two
arguments as parameters. The two contexts will both use the {\tt
  push} procedure, one using the arguments {\tt 1} and {\tt 1} and the
other {\tt 2} and {\tt 1}.


\begin{lstlisting}
int main() {

  char stack1[8*1024];
  char stack2[8*1024];

  /* The first context. */
  
  getcontext(&cntx_one);
  cntx_one.uc_stack.ss_sp = stack1;
  cntx_one.uc_stack.ss_size = sizeof stack1;
  makecontext(&cntx_one, (void (*) (void)) push, 2, 1, 1);

  getcontext(&cntx_two);
  cntx_two.uc_stack.ss_sp = stack2;
  cntx_two.uc_stack.ss_size = sizeof stack2;
  makecontext(&cntx_two, (void (*) (void)) push, 2, 2, 1);

  running = 1;
  
  swapcontext(&cntx_main, &cntx_one);

  return 0;
}
\end{lstlisting}

When you compile and run the program you might not be super exited
since nothing much happens. We start the program and then pass
control over to context one. This context will of course call {\tt
  push()} and push and pop the stack before terminating but the
second context is never given time to execute. However, everything is
prepared to swap between the contexts. The only thing we need to do is
add a call to {\tt yield()} in the {\tt push()} procedure. Let's call
yield when were at the top of the stack.


\begin{lstlisting}
     :
  } else {
    printf("%d%*s top\n", p, i, " ");
    yield();
  }
}
\end{lstlisting}

Now things are starting to look more interesting; the two contexts
take turn executing. You can do a yield in every recursion if you like
and see how the execution jumps from one context to the other. Before
we're done we shall fix one more thing.

\section{termination}

The switching between contexts works fine but there is one problem
with our example that might not be obvious. When one context
terminates the whole execution terminates. If we yield the execution
when at the top of the stack we will see that only the first context
will finish the execution. The first context will push all entries on
its stack, then the second context will do the same but then when the
first context resumes execution it will pop all entries and terminate
the whole program. We need a way to capture the termination and
schedule the second context again.

The solution that we will implement is not the most general
solution. It is, as the {\tt yield()} procedure, hardwired into only
handle our two contexts. We will later discuss a more general solution
but this will do for now.

We begin by adding yet another context, {\tt cntx\_done}, this will be the context that
we switch to when a context terminates. We also add two flags that
will keep track of which contexts that have terminated their execution.

\begin{lstlisting}
static int done1;
static int done2;

static ucontext_t cntx_done;
\end{lstlisting}

Next we define a procedure {\tt done()} that will be the procedure of
{\tt cntx\_done}. This procedure will be scheduled when one of the
contexts terminate and we should then switch to the remaining but we
need to keep track of which context is still alive so we do some
bookkeeping using the flags {\tt done1}, {\tt done2} and {\tt
  done}. When both contexts have terminated we're done.

\begin{lstlisting}
void done() {

  int done = 0;

  while(!done) {
    if(running == 1) {
      printf(" - process one terminating -\n");
      done1 = 1;
      if(!done2) {
	running = 2;
	swapcontext(&cntx_done, &cntx_two);
      } else {
	done = 1;
      }
    } else {
      printf(" - process two terminating -\n");
      done2 = 1;
      if(!done1) {
	running = 1;
	swapcontext(&cntx_done, &cntx_one);
      } else {
	done = 1;
      }
    }
  }
  printf(" - done terminating -\n");
}
\end{lstlisting}

Now for the part where we set this up. We need another stack for {\tt
  cntx\_done} so we allocate it as the other stacks. Next we do the
trick that will make things work, we set a element called {\tt
  uc\_link} in each of the contexts {\tt cntx\_one} and {\tt
  cntx\_two}. This should be done before the call to {\tt
  makecontext()}. This will set u the context in such way that when
the context terminates we will automatically switch to the context{\tt
  cntx\_done}.
\begin{lstlisting}
    :
  char stack_done[8*1024];
    :

  cntx_one.uc_link = &cntx_done;
    :

  cntx_two.uc_link = &cntx_done;
    :
\end{lstlisting}

The last thing we need to do is initialize the new context. We here
also give it the context {\tt cntx\_main} to be the context that it
should switch to when terminating.

\begin{lstlisting}
  getcontext(&cntx_done);
  cntx_done.uc_link = &cntx_main;
  cntx_done.uc_stack.ss_sp = stack_done;
  cntx_done.uc_stack.ss_size = sizeof stack_done;
  makecontext(&cntx_done, (void (*) (void)) done, 0);  
    :  
\end{lstlisting}

We still start the execution by switching to the first context but we
add some printout to follow the execution. Note that when we switch to
the first context we store the current execution environment in {\tt
  cntx\_main}. This is the context that we switch to once the {\tt
  cntx\_done} is terminating. We will if everything works out see the
last printout before the whole execution terminates.

\begin{lstlisting}
  running = 1;

  printf(" - let's go! \n");  
  swapcontext(&cntx_main, &cntx_one);
  printf(" - that's all folks -\n");
\end{lstlisting}

If everything works you should now see how the two contexts both
execute to termination. Add a call to {\tt yield()} at several places
in the procedure {\tt push()} to switch several times, it works right?

\section{a library}

Before you started this assignment you probably had no idea of how to
implement your own threads library. Now it should be, at least partly,
clear on how to start. The experiment that we have done has actually
done the hard part, switching between different context. Tet has of
course been hardwired into only work for the two contexts that we have
used but we could generalize it to work for several contexts, how hard could it be?

We would need to keep track of several contexts so why not keep a list
of contexts that should be scheduled. When a context calls {\tt
  yield()} we would simply place the context last and schedule the
next context in the list.

If a context terminated, the {\em done context} would be scheduled and
it would deallocate the context and schedule the next context in the
list. If the list is empty the execution terminates. We would of
course have to provide a procedure {\tt spawn()} that would allocate a
new context on the heap (with it's own stack) and insert it in the list
of contexts to schedule.

We could continue to rely on explicit yielding of execution but we
could also add a timer interrupt that would make a call to {\tt
  yield()} every 100 ms. Adding support to wait for a context to
terminate could be one feature but would require some bookkeeping; we
need to attach the context to a list of suspended threads attached to
the thread that we are waiting for. There are probably more features
that we would like to have but nothing that is too hard to solve. 

You might wonder why one would like to implement threads library but
it could have its advantages. The switching between contexts could be
more efficient since the kernel is not involved. There are however many
things that could go wrong so one should have very good reasons for
not using the standard pthreads library. The idea with this assignment
is not to make you implement a new threads library but to get some
insight into threads scheduling and that it's not all magic.






\end{document}
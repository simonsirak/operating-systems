\documentclass[a4paper,11pt]{article}

\usepackage[T1]{fontenc}
\usepackage[latin1]{inputenc}

\usepackage{listings} \lstset{language=c, showstringspaces=false, xleftmargin=-60pt, frame=l}

\usepackage{graphicx}


\newcommand*\justify{%
  \fontdimen2\font=0.4em% interword space
  \fontdimen3\font=0.2em% interword stretch
  \fontdimen4\font=0.1em% interword shrink
  \fontdimen7\font=0.1em% extra space
  \hyphenchar\font=`\-% allowing hyphenation
}
\begin{document}

\begin{center} \vspace{20pt} \textbf{\large Ping-pong and Echoes}\\
  \vspace{10pt} \textbf{Johan Montelius}\\ \vspace{10pt} \textbf{HT2016}
\end{center}

\section{Introduction}

In this tutorial you're going to explore different ways for two
processes to communicate with each other. We will first look at {\em
  pipes}, something that you have probably used when working in the
shell but now we will implement our own pipes. We will then look at
{\em sockets}, an abstraction that can be used for process
communication even if the processes reside on different machines in a
network.



\section{Pipes}

To understand pipes you only have to understand how to read and write
to streams such as standard input or standard output. A stream is of
course nothing else but a sequence of bytes and we have already seen
how {\tt read()} and {\tt write()} can be used to operate on a file
descriptor.

When we create a pipe we will be given two file descriptors, one that
will be used by the producer and one that can be used by a consumer. If
we want to create two processes that can communicate of a pipe, we
first create the pipe and then {\tt fork()} two processes. Since any
forked process will share the same file descriptor table, we can set
up one process to use the {\tt write descriptor} to produce data and
the other to use the {\em read descriptor} to consume data.

\subsection{flow control}

The nice thing with the pipe is that it will take care of {\tt flow
  control} i.e. the producer will be suspended if the consumer can not
process the information quick enough. Let's run an experiment to see
this in action, open a file called {\tt flow.c} and add the following code. 

We will let a producer send ten bytes in a burst and do this ten
times. We will later change these numbers so let's define them as
macros.

\begin{lstlisting}
  #include <stdlib.h>
  #include <stdio.h>
  #include <unistd.h>
  #include <assert.h>
  #include <sys/wait.h>

  #define ITERATIONS 10
  #define BURST 10
\end{lstlisting}

In the main procedure we first create a pipe providing an array where
the procedure will store the two file descriptors. We then fork a new
process determine if we're the mother or child process.


\begin{lstlisting}
  int main() {
    int descr[2];

    assert(0 == pipe(descr)); 

    if(fork() == 0 ) {
      /* consumer */
      :
    }

    /* producer  */
    :  
    wait(NULL);
    printf("all done\n");
  }
\end{lstlisting}

The consumer will read the burst of ten bytes each but will sleep for
a second in between bursts. This is to simulate a process that is
receiving some data and then spend some time processing this data.

\begin{lstlisting}
  for(int i = 0; i < ITERATIONS; i++) {
    for(int j = 0; j < BURST; j++) {
      int buffer[10];
      read(descr[0], &buffer, 10);
    }
    sleep(1);
  }
  printf("consumer done\n");
  return 0;
  
\end{lstlisting}

The procedure will be more eager and will produce data as quickly as
possible. To keep track of the pace we let it print a message by the
end of each burst.

\begin{lstlisting}
  for(int i = 0; i < ITERATIONS; i++) {
    for(int j = 0; j < BURST; j++) {
      write(descr[1], "0123456789", 10);
    }
    printf("producer burst %d done\n", i);
  }
  printf("producer done\n");
\end{lstlisting}

That is it, compile and run the benchmark - explain what is
happening. Now increase the size of the burst to a hundred, to a
thousand - what is happening? What you see is an effect of {\em flow
  control}, something very important when it comes to any type of
communication.

\subsection{named pipes}

Pipes are very easy to use since the only thing we need to understand
is how to read a sequence of bytes from a file descriptor. They are
also very easy to set up ... if you're forking a new process. The
question is how to do it if you're not the one creating the new
process. 

It turns out that we can use the same naming scheme as for files to
register and find pipes. We can thus create a pipe and register it
under a file name. To do this we use the library call {\tt mkfifo()} -
check the man pages, note that it exists both as a command and as a
library call (use {\tt >man 3 mkfifo}). 

We need half a dozen of include statements so create two files, {\tt
  cave.c} and {\tt baba.c}. They will look very much like the {\tt
  flow.c} but now we split the producer (the cave) from the consumer
({\em Ali Baba}), and connect them through a pipe called {\tt sesame}. 

\begin{lstlisting}
  #include <stdlib.h>
  #include <stdio.h>
  #include <unistd.h>
  #include <assert.h>
  #include <sys/wait.h>
  #include <sys/types.h>
  #include <sys/stat.h>
  #include <fcntl.h>


  #define ITERATIONS 10
  #define BURST 1000
\end{lstlisting}

In {\tt cave.c} we have the producer and also the process that will
create the pipe. This is were we call {\tt mkfifo()} providing the
name {\tt sesame} and the {\em mode} of the pipe. Once the pipe is
created we will {\em open} the pipe using a call {\tt open()}. This is
done in the same way as we would have opened any file. In this example
we open it in write mode only since the producer will only write to
the pipe. The rest of the file uses the producer part of {\tt flow.c}
i.e. looping over the {\tt ITERATIONS} and {\tt BURST}, the only
difference is that we now write to {\tt pipe}.

\begin{lstlisting}
  int main() {
    /* create the named pipe */
    int mode = S_IRUSR | S_IWUSR | S_IRGRP | S_IROTH;
    mkfifo("sesame", mode);  

    int flag = O_WRONLY;  
    int pipe = open("sesame", flag);
    :  
    :
    printf(``producer done\n'');
    return 0;
  }
\end{lstlisting}

The consumer assumes that there is a pipe called {\tt sesame} and will
open this for reading. Complete the code by using the consumer part
from {\tt flow.c}.

\begin{lstlisting}
  int main() {
    /* open pipe for reading */
    int flag = O_RDONLY;  
    int pipe = open("sesame", flag);
    :
    :
    printf("consumer done\n");
    return 0;
  }
\end{lstlisting}

Compile and run the producer and consumer in the same shell, use the
{\tt \&} notation to set a program to run in the background.

\begin{verbatim}
./cave&
[cave] 15321
> ./baba 
producer burst 0 done
producer burst 1 done
   :
   :
\end{verbatim}

Look at the man pages for {\tt open()} (use {\tt >man 3 open}).
implicitly say that the file should not be created if it does not exist
i.e. the call will suspend until we create the pipe. We also implicitly
say that the reader should suspend waiting for the writer and also
that the writer should wait for a reader (check {\tt O\_NONBLOCK}).
This means that we equally well can start the producer before the
consumer.

\begin{verbatim}
./baba&
[baba] 15351
> ./cave 
producer burst 0 done
producer burst 1 done
   :
   :
\end{verbatim}

\subsection{marshaling}

The easy way in which we can use pipes to communicate has a
down-side. How do you send such a simple thing as a {\em floating-point} from one
process to the other? The problem is of course that we're not sending
data structures but bytes in the pipe. This means that we have to send
the bytes across and then interpret them as a floating-point on the other
side.

Let's try to send some floating-points from one process to the
other. Make a copy of {\tt flow.c} and call it {\tt marshal.c}. Then
do the following changes, the consumer part will look like this:

\begin{lstlisting}
  for(int i = 0; i < ITERATIONS; i++) {
    double buffer;
    read(descr[0], &buffer, sizeof(double));
    printf("received %f\n", buffer);
    sleep(1);
  }
\end{lstlisting}

and the producer part will look like this:

\begin{lstlisting}
  for(int i = 0; i < ITERATIONS; i++) {
    double pi = 3.14*i;
    write(descr[1], &pi, sizeof(double));
  }
\end{lstlisting}

Try this out and see that it works. 

The reason why it works is of course because the producers and
consumer represents floating-point data structures in the same way. Note
that we have no clue of what this representation looks like but we
know that if we send some bytes across a pipe and interpret them as
the data structure that we know they represent then obviously it will
work. 

The above is not necessarily true if the processes were running on
different machines, were written in different languages or even
compiled with different versions of a compiler. We can use this simple
way of marshaling data structures since we know that both the
producer and the consumer agree on how things are represented.

Things do however become very complicated if we for example want to
send a linked list across the pipe. Obviously you can not send a
memory reference across a pipe so you would have to come up with your
own representation of linked data structures.

The problem of turning data structures into sequences of bytes is
called {\em marshaling}. We will not explore this further but you
should realize that it is a problem.

\section{Sockets}

Pipes are a very efficient way of communicating between processes but
it has one draw back - it is only one-way communication. In order for
two processes to have a two-way communication you would have to use
two pipes, one in each direction. This is of course easily done but it
would be nice to have an abstraction that provided two-way
communication directly - introducing {\em sockets}.


A {\em socket} is the abstraction of pipe communication were we do not
assume a shared file system. We should be able to open up a
communication channel with any process even if it is not running on
the same machine. If the processes are actually running on the same
machine, things will be as efficient as using pipes but the user level
application do not need to be aware of how things are implemented.

You should already be familiar with network communication so we will
not go through TCP and UDP but rather take a look at sockets from an
applications point of view.

\subsection{ping-pong}

When we create a socket we choose the {\em domain} for the
socket. Look up the different domains using {\tt man socket}, as you
can see the domains specify the networking protocol that the socket
should use. There is however one domain that is a bit different, the
{\tt AF\_UNIX} domain. This domain is used if we know that the
processes are running on the same machine. 

Let's implement a {\em ping-pong} experiment where two processes are
sending ping messages to and from. Create two files {\tt ping.c} and
{\tt pong.c}. The two processes will look very similar but {\tt ping}
will serve as the server and {\tt pong} will act as a client i.e. it
will find and connect to {\tt ping}.

We need some header files and among the usual suspects we find {\tt
  socket.h} that will provide the support for sockets and {\tt un.h}
that is the special {\tt AF\_UNIX} domain. 

\begin{lstlisting}
#include <stdio.h>
#include <stdlib.h>
#include <unistd.h>
#include <sys/socket.h>
#include <sys/un.h>
#include <assert.h>

#define NAME "pingpong"
#define TURNS 10
\end{lstlisting}

The name {\tt pingpong} is the address that we will use to register
the socket. As with pipes we will use the regular file system to
provide name resolution.

In both files we create a socket and a socket address {\tt \{AF\_UNIX,
  NAME\}}.

\begin{lstlisting}
int main(void) {
  int sock;

  assert((sock = socket(AF_UNIX, SOCK_STREAM, 0)) != -1);

  struct sockaddr_un name = {AF_UNIX, NAME};  
\end{lstlisting}

In {\tt ping.c} we then {\em bind} the socket to the address i.e. we
create the {\tt pingpong} name in the file system in order for the
{\tt pong} process to find it. We also start to {\em listen} to the
socket and thereby tell the operating system that we are prepared to
accept connecting clients. The call to {\tt assert()} will suspend
until a client connects and then return a new socket descriptor {\tt
  cfd}. This is our two-way connection that we can use for
communication. 

\begin{lstlisting}
  assert(bind(sock, (struct sockaddr *)&name, sizeof(name)) != -1);

  assert(listen(sock, 5) != -1);

  int cfd;
  
  assert((cfd = accept(sock, NULL,  NULL)) != -1);
\end{lstlisting}

The file {\tt pong.c} looks very similar but now we do not have to
bind the socket nor wait for incoming connection. We simply us the
socket and the socket address to connect to the server.

\begin{lstlisting}
   assert(connect(sock, (struct sockaddr *)&name, sizeof(name)) != -1);
\end{lstlisting}

The call to {\tt connect()} will only return a error if something goes
wrong but it will also connect the socket so we can use it to
communicate with the {\tt ping} process.

Both the {\tt ping.c} and {\tt pong.c} files then contain a section
were we use {\tt send()} and {\tt recv()} to pass messages to a
fro. Below is the code for {\tt ping.c} and you will have to adapt it
for {\tt pong.c}. 

\begin{lstlisting}
  for(int i = 0; i < TURNS; i++) {
    char buffer[5];
    assert(send(cfd, "ping", 4, 0) != -1);
    assert(recv(cfd, buffer, 4, 0) != -1);
    buffer[4] = 0;
    printf("ping received %s\n", buffer);
  }
\end{lstlisting}

The file {\tt ping.c} will of course look very similar, only
difference is that we will use {\tt sock} when we communicate and
that we will first receive a message before sending a message.

If everything compiles you should be able to run the experiment as follows:

\begin{verbatim} 
> ./ping &
[ping] 4427
> ./pong
ping received ping
pong received ping
ping received ping
   :
\end{verbatim}

If you run it a second time you will receive the strange message:

\begin{verbatim}
ping: ping.c:21: main: Assertion `bind(sock,  .....
aborted (core dumped)
\end{verbatim}

If you look in your directory using {\tt ls -l} you will see the file
name {\tt pingpong} with the strange description {\tt srwxrwxr-x}. The
{\tt s} means that this is not a regular file but rather a name of a
socket. When we try to bind the socket to a name we get an error since
the name is already take. To prevent this we should {\tt unlink()} the
name when we're done. Remove the file and add {\tt unlink(NAME)} to
the end of {\tt ping.c}. 

You also see that people add a call to {\tt unlink()} before doing a
call to {\tt bind()} but then one should of course be sure, that no one
is actually using the name.

\subsection{datagrams}

One problem with the solution that we have so far is that we want to
send messages at the application layer but the communication layer only
provides an abstraction of a stream of bytes. In our program we have
solved this problem by knowing that a message is four bytes. It's a
bit more complicated if messages can be of arbitrary size. 

We could, and many communication protocols work this way (http for
example), encode the length of a message in the first byte (or two) and
then read as many bytes as required. However, since the problem is so
often encountered there is an abstraction that provides what we're
looking for - {\em datagrams}.

Assume that we want to send text strings to a server and we want the
server to convert the string to lower case and then send it back. We
can implement this quite easily using the datagram socket type. Create
two files {\tt tolower.c} and {\tt conv.c}. They will look very similar
so we will go through them and point out the differences. 

We will of course use some header files but by now you should be able to
figure out which ones to use using the {\tt man} command. We start by
showing the main procedure.

\begin{lstlisting}
#define SERVER "lower"
#define MAX 512

int main(void)  {
  int sock;
  char buffer[MAX];

  /* A socket is created */
  assert((sock = socket(AF_UNIX, SOCK_DGRAM, 0)) != -1);

  struct sockaddr_un name = {AF_UNIX, SERVER};

  assert(bind(sock, (struct sockaddr *) &name, sizeof(struct sockaddr_un)) != -1);
       :  
\end{lstlisting}

The server, {\tt tolower.c}, will as before create a socket but now we
state that it is of {\tt SOCK\_DGRAM} type. As before we register it
under a name in order for the client to find us. 

The client, {\tt conv.c}, will look the same but we register the
socket under a name of its own. Note that we will later use the {\tt
  SERVER} name so keep this in the file. We also provide a text to
test the service.

\begin{lstlisting}
#define TEST "This is a TeSt to SEE if iT wORks"
#define CLIENT "help"

int main(void) {
     :
    struct sockaddr_un name = {AF_UNIX, CLIENT};
     :
\end{lstlisting}

It is now time for the server to wait for incoming messages and we do
this using the socket directly i.e. there is no call to {\tt
  listen()}. We create a data structure to hold the address of the
client that connects and then call {\tt recvfrom()}. When this call
returns we will have a message in the buffer that we provided - since
it is a datagram service we will have the whole message.

\begin{lstlisting}
    struct sockaddr_un client;
    int size = sizeof(struct sockaddr_un);
    
    while(1) {
      int n;
      n = recvfrom(sock, buffer, MAX-1, 0, (struct sockaddr *) &client, &size);
      if (n == -1)
        perror("server");

      buffer[n] = 0;
      printf("Server received: %s\n",buffer);
	
      for (int i= 0; i < n; i++)
         buffer[i] = tolower((unsigned char) buffer[i]);

      assert(sendto(sock, buffer, n, 0, (struct sockaddr *) &client, size) == n);
    }
\end{lstlisting}

The server will convert the message to lower case letters and then
send it back to the client. Notice that we know who we should send it
to since we received the address of the client in the call to {\tt
  recvfrom()}. There is no established connection between the two
processes, the two messages are independent from each other.

On the client side we have a similar scenario but here we know who we
want to send the message to. This is where we need the name {\tt
  SERVER}, in our call to {\tt sendto()} we must provide the address
of the server.

\begin{lstlisting}
    struct sockaddr_un server = {AF_UNIX, SERVER};
    int size = sizeof(struct sockaddr_un);
    
    int n = sizeof(TEST);

    assert(sendto(sock, TEST, n, 0, (struct sockaddr *) &server, size) != -1);
    n = recvfrom(sock, buffer, MAX-1, 0, (struct sockaddr *) &server, &size);
    if (n == -1)
      perror("server");

    buffer[n] = 0;
    printf("Client received: %s\n", buffer);
    unlink(CLIENT);
    exit(0);
}
  
\end{lstlisting}

Hard coding the address of the client might not be a very good idea if
we want to run multiple instances of the client. The names must of
course not collide and we would be better of with selecting the
addresses dynamically. 

\subsection{sequence of datagrams}

The beauty of datagrams is that they are independent from each other
and this is also the problem. We are not guaranteed that messages are
delivered in order nor will we be informed if a datagram is lost.

This will of course not happen if we have the two processes running in
the same operating system but it is a scenario if we run across the
Internet. If this is the case we might want to use the {\tt
  SOCK\_SEQPACKET} that will give us a reliable ordered delivery of
messages. 

\subsection{address families}

We have seen how sockets can use the file system as the address but
there are many different {\em address families} that we can use. The
most used is the {\em internet protocol} that will allow us to
communicate between two processes on different machines in a
network. The socket abstraction allows us to change the underlying
communication protocol and allows the application level to remain the
same. 


To demonstrate how easy it is to set up two processes that communicate
over the network we will implement an {\em echo server}, a process that
simply accepts incoming messages and then return them to their
destination. The implementation will look very much like the {\tt
  tolower.c} and {\tt conv.c} so you can use these files as templates
for two files called {\tt echo.c} and {\tt hello.c}.

Both files need the same set of header files, notice the {\tt
  netinet/ip.h} and {\tt arpa/inet.h} that will give us support for the
IP sockets and some address translations that will come in handy.

\begin{lstlisting}
#include <stdlib.h>
#include <stdio.h>
#include <sys/socket.h>
#include <netinet/ip.h>
#include <arpa/inet.h>
#include <assert.h>
\end{lstlisting}

The echo process will create a socket and bind it to the {\em port}
{\tt 8080}. The procedure {\tt htons()} and {\tt htonl()} will convert
short (16-bit) and long (32-bit) integers to {\em network byte order}
i.e. the order the two or four bytes should appear when we send them to
be interpreted correctly by the network infrastructure. 

We (I assume that you only have one network connection) do not specify
the ip-address but allow the operating system ({\tt INADDR\_ANY}) to
choose this for us.

\begin{lstlisting}

#define PORT 8080
#define MAX 512

int main(void)  {
    int sock;
    char buffer[MAX];

    /* A socket is created */
    assert((sock = socket(AF_INET, SOCK_DGRAM, 0)) != -1);

    struct sockaddr_in name;
    
    name.sin_family = AF_INET;
    name.sin_port = htons(PORT);
    name.sin_addr.s_addr = htonl(INADDR_ANY);

    assert(bind(sock, (struct sockaddr *) &name, sizeof(name)) != -1);
       :  
\end{lstlisting}

We then do exactly as what we did in {\tt tolower.c} but we do not
convert the string to lower-case before sending it back.  We can add
another print out statement to see who we got the message from. The
procedure {\tt inet\_ntoa()} will take a network address and turn it
into a string and {\tt ntohs()} will take a short network address and
turn it into an integer.

\begin{lstlisting}
  printf("Server: received: %s\n", buffer);
  printf("Server: from destination %s %d\n", inet_ntoa(client.sin_addr), ntohs(client.sin_port));
\end{lstlisting}

That's it for the echo server, the client {\tt hello.c} will be almost
as simple. We include the same header files but now of course need the
address of the server in order to know how to connect. You can specify
the server ip-address as a string as shown in the example below but
there are other possibilities. In this example we use the {\em
  loop-back address 127.0.0.1} i.e. the machine it self.


\begin{lstlisting}
#define MAX 512
#define TEST "Hello, hello"

#define SERVER_IP "127.0.0.1"
#define SERVER_PORT 8080
\end{lstlisting}

The hello process of course needs a socket by its own and bind this
socket to an address in order for the server to send back a reply. In
this case we allow that operating system to select a port number for us
since we will only use this for reply messages, we do not need to
use a known port number.

\begin{lstlisting}
    struct sockaddr_in name;

    name.sin_family = AF_INET;
    name.sin_port = 0;
    name.sin_addr.s_addr = htonl(INADDR_ANY);

    assert(bind(sock, (struct sockaddr *) &name, sizeof(name)) != -1);    
\end{lstlisting}

We can now create the address of the server, this is were we need the
server port number and server ip-address.

\begin{lstlisting}
    struct sockaddr_in server;

    server.sin_family = AF_INET;
    server.sin_port = htons(SERVER_PORT);
    server.sin_addr.s_addr = inet_addr(SERVER_IP);  
\end{lstlisting}

This is it, we now do exactly what we did in the file {\tt conv.c}
i.e. send a message and wait for the reply. If everything works you
should be able to start the echo server and then run {\tt hello} from
your own machine. If you check you ip-address and change the {\tt
  hello.c} you should be able to do the same thing across the network. 


\subsection{Alternatives}

One alternative mechanism for process communication is the concept of
{\em signals}. Signals are however not intended as a tool for process
communication, rather a way for the operating system to alert the
process of events and possibly for a process to control the faith of
another process i.e. kill it or suspend its execution. Using signals
for communication is not advisable since it is likely to introduce
more problems that it will actually solve.

Another concept is the use of {\em shared memory}. We have already
seen how two threads in a process can collaborate when they share the
same memory and it is possible for two processes to do almost the
same. One process can map a file into memory to provide direct access
to its content and if two processes map the same file they will share
this content. It is however very different from the memory shared by
two threads since the file is mapped into different virtual memory
regions. To choose this approach you really have to know what you're
doing.

The socket abstraction layer is what everything on the Internet is
built on top of. It might however not be the best things to use when
you're building an application that will run on a single operating
system. You would like to have more support from a communication layer
to handle for example {\tt authentication}, {\tt multicasting} and
{\em publish subscribe} functionality. There are many messaging
abstraction layers that provide this but to explore these require
another tutorial. If you want to explore something by yourself that is
highly related to operating systems, take a look at {\em D-Bus}.

\section{Summary}

Pipes, a simple way to use communication abstraction that used the notion
of reading and writing to file descriptors in the same way as you
would access a file.

Sockets provide an abstraction on a higher level that provides two-way
communication using either byte streams, messages, sequences of
messages etc. We have several networking protocols to choose from and
we can change the choice of {\em address family} without changing the
main part of our application.



\end{document}

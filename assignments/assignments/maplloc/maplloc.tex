\documentclass[a4paper,11pt]{article}

\usepackage[T1]{fontenc} 
\usepackage[latin1]{inputenc}
 
\usepackage{listings} \lstset{language=c, breaklines=true, showstringspaces=false, xleftmargin=-10pt, frame=l}

\usepackage{graphicx}


\newcommand*\justify{%
  \fontdimen2\font=0.4em% interword space
  \fontdimen3\font=0.2em% interword stretch
  \fontdimen4\font=0.1em% interword shrink
  \fontdimen7\font=0.1em% extra space
  \hyphenchar\font=`\-% allowing hyphenation
}
\begin{document}

\begin{center} \vspace{20pt} \textbf{\large My malloc using mmap}\\
\vspace{10pt} \textbf{Johan Montelius}\\ \vspace{10pt} \textbf{HT2016}
\end{center}


\section{Introduction}

This is a version of the Mylloc experiment but here we will use the
system call {\tt mmap()} instead of the system call {\tt sbrk()} when
we're asking for memory from the operating system. If you're running
on an OSX machine this is probably what you want to do since {\tt
  sbrk()} is deprecated and if it is used,it is implemented in
terms of {\tt mmap()}. It can also be interesting to look at even if
you have completed the first assignment using {\tt sbrk()} since you will 
learn how to use {\tt mmap()} to do what we want. 

You need to read this tutorial together with the original
tutorial. In this short tutorial we will implement our own version of
{\tt sbrk()} so that the rest of the Mylloc system can run without any
modifications. If one would do a {\tt mmap()} solution from scratch
this is probably not the most straight forward solution.

\section{Implement sbrk()}

We will implement the first simple version of malloc as described in
the Mylloc paper. What we need is therefore an implementation of {\tt
  sbrk()}, if we can fake this we're home. Create a file called {\tt
  sbrk.c} and start coding.

In the first part of {\tt sbrk.c} we use a GCC extension to the C
language. We're declaring a function, {\tt init()}, to be a {\em
  constructor}. This means that the function will be called when the
program is loaded. The feature is normally used by shared libraries
but we will make use of it in our experiment (does this work using
Clang?).


\begin{lstlisting}
#include <stdlib.h>
#include <sys/mman.h>

#define MAX_HEAP 64*1024*4096

char *heap;

char *brkp = NULL;
char *endp = NULL;

static void init() __attribute__((constructor));

void init() {
  heap = (char *)mmap(NULL, MAX_HEAP,  
                      (PROT_READ | PROT_WRITE), 
                      (MAP_PRIVATE | MAP_ANONYMOUS), -1, 0);
  brkp = heap;
  endp = brkp + MAX_HEAP;
}
\end{lstlisting}

What we will do in the initialization is to allocate a heap of our
own. We do this with a call to the {\tt mmap()} system call. The
procedure will allocate a huge area ({\tt MAX\_HEAP} big) that is read
and writable. It is a private area and is not the mapping of a file
i.e. simply new virtual memory.

Once we know this we can implement a procedure {\tt sbrk()} that when
requested will increment the {\tt brkp} pointer and return a pointer
to the start of the allocated block. 

\begin{lstlisting}
void *sbrk(size_t size) {
  if(size == 0) {
    return (void *)brkp;
  }
  void *free = (void *)brkp;

  brkp += size;
  if(brkp >= endp) {
    return NULL;
  }
  return free;
}
\end{lstlisting}

We know that {\tt sbrk()} should return the {\em top of the heap} when
called with a zero argument so this is what we do. If we have reached
the end of the heap we return {\tt NULL} which is also fine according
to the documentation. We could of course do another call to {\tt
  mmap()} to get more memory but this is fine for our purposes.

If you now compile this file we will have an object file that implements {\tt sbrk()}. 

\begin{verbatim}
 > gcc -c sbrk.c
\end{verbatim}

We can then link everything together and doing so the call to {\tt
  sbrk()} in {\tt mylloc.c} will use our version.

\begin{verbatim}
 > gcc -o bench sbrk.o rand.o mylloc.o bench.c -lm
\end{verbatim}

If you look carefully you will notice that our mmapped heap is not
located at the same place as the regular heap.

\end{document}
